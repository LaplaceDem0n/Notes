\documentclass[17pt,UTF-8,a4paper]{ctexart}
%\usepackage[]{ctex}
%\usepackage[explicit]{titlesec}

\CTEXoptions[today=old]
%日期使用英文格式

%\renewcommand\thesection{1.\arabic{section}}
%暂时通过每次作业修改这个地方和标题来使得作业与题目Match。




\usepackage{fancyhdr}
\usepackage{extramarks}
\usepackage{amsmath}%\usepackage{amsthm}
%\usepackage{amsfonts}
%\usepackage{tikz}
%\usepackage[plain]{algorithm}
%\usepackage{algpseudocode}

%\usetikzlibrary{automata,positioning}

%
% Basic Document Settings
%

\topmargin=-0.45in
\evensidemargin=0in
\oddsidemargin=0in
\textwidth=6.5in
\textheight=9.0in
\headsep=0.25in

\linespread{1.1}

\pagestyle{fancy}
\lhead{\hmwkAuthorName}
\chead{\hmwkClass\ (\hmwkClassInstructor): \hmwkTitle}
%\chead{\hmwkClass\ (\hmwkClassInstructor\ \hmwkClassTime): \hmwkTitle}
\rhead{\firstxmark}
\lfoot{\lastxmark}
\cfoot{\thepage}

\renewcommand\headrulewidth{0.4pt}
\renewcommand\footrulewidth{0.4pt}

\setlength\parindent{0pt}

%\setcounter{secnumdepth}{2}
\newcounter{partCounter}
%\newcounter{homeworkProblemCounter}
%\setcounter{homeworkProblemCounter}{0}
%\nobreak\extramarks{Problem \arabic{homeworkProblemCounter}}{}\nobreak{}


%
% Homework Details
%   - Title
%   - Due date
%   - Class
%   - Section/Time
%   - Instructor
%   - Author
%

\newcommand{\hmwkTitle}{Problem\ Set\ 1}
\newcommand{\hmwkDueDate}{\today}
\newcommand{\hmwkClass}{Graph Theory}
\newcommand{\hmwkClassTime}{}
\newcommand{\hmwkClassInstructor}{程龚}
\newcommand{\hmwkAuthorName}{\textbf{Name}\:彭自远\quad\textbf{ID}\:171180631}
%\newcommand{\hmwkAuthorName}{\textbf{彭自远} \and \textbf{Davis Josh}}

%
% Title Page
%

\title{
    \vspace{2in}
    \textmd{\textbf{\hmwkClass:\ \hmwkTitle}}\\
    \normalsize\vspace{0.1in}\small{Printed\ on\ \hmwkDueDate}\\
    \vspace{0.1in}\large{\textit{\hmwkClassInstructor}}
    %\vspace{0.1in}\large{\textit{\hmwkClassInstructor\ \hmwkClassTime}}
    \vspace{3in}
}

\author{\hmwkAuthorName}
\date{}

\renewcommand{\part}[1]{\textbf{\large Part \Alph{partCounter}}\stepcounter{partCounter}\\}

%
% Various Helper Commands
%

% Useful for algorithms
\newcommand{\alg}[1]{\textsc{\bfseries \footnotesize #1}}

% For derivatives
\newcommand{\deriv}[1]{\frac{\mathrm{d}}{\mathrm{d}x} (#1)}

% For partial derivatives
\newcommand{\pderiv}[2]{\frac{\partial}{\partial #1} (#2)}

% Integral dx
\newcommand{\dx}{\mathrm{d}x}

% Alias for the Solution section header
\newcommand{\solution}{\textbf{\large Solution}}

% Probability commands: Expectation, Variance, Covariance, Bias
\newcommand{\E}{\mathrm{E}}
\newcommand{\Var}{\mathrm{Var}}
\newcommand{\Cov}{\mathrm{Cov}}
\newcommand{\Bias}{\mathrm{Bias}}

\begin{document}

%\maketitle

%\pagebreak

\section*{Problem Set 1}

\subsection*{Problem 1.4 度}
\paragraph*{反证法}
假设在一个n阶简单图中,没有任意两个点度数相同。

\paragraph*{}
在一个有n个顶点的图中,点的度一定在0到n-1之间,共有n种可能。如果有度为0的顶点,则余下n-1个顶点的度的取值范围为1到n-2之间(如果为0则有两个点度数相同),共有n-2种可能,由鸽笼原理知必有两个点度数相同。
如果没有度为0的顶点,则n个顶点的度的取值范围为1到n-1,共有n-1种可能,同样由鸽笼原理知必有两个点度数相同。

\paragraph*{}
综上,在一个n阶简单图中,至少有两个点度数相同。


\subsection*{Problem 1.35 同构}




\paragraph*{结论}
左图与中间的图同构。左图和中间的图都与右图不同构。

\paragraph*{证明 Part 1}
左图与中间的图的双射如下:\\
左图顶点:d g e b h c a f\\
中图顶点:u v t s z y x w 

\paragraph*{证明 Part 2}
中间的图为二部图,而右图非二部图。\\
另外,由于左图与中间的图同构,中间的图与右图不同构,左图与右图也不同构。

\subsection*{Problem 1.23 长度} 
\paragraph*{证明 Part 1}
对图$G$中任意点$v_0$,删去该点后,$\delta({G-v_0})\geq{k-1}$。\\
在图的最小度不为1时,总能接着删去新的点。\\  
不断地删去点直到得到图的最小度为0。共删去了k个相邻的点,这k个相邻点即可组成长度为k的路。


\paragraph*{证明 Part 2}
设图$G$中度最小的点为$v_0$,设图中最长路为:$P=v_0,\dots,v_k$。此时图内最小圈的长度取决与于$v_0$相邻的在路径P上的最远点的位置,圈长度的最小情况出现在$v_1,v_2,\dots,v_{\delta(G)}$都与$v_0$相邻时。此时,$v_0,v_1,v_2,\dots,v_{\delta(G)},v_0$为最小圈。
%当$\Theda(G)$

\subsection*{Problem 1.31 连通}
\paragraph*{证明}
由于图$G$是非连通的,如果图中只有两个顶点,结论易证。\\
否则,对图$G$中有边相连的点$u$和$v$,总可以找到另一连通分量的点$w$,使得在图$\overline{G}$中,$u$与$w$间存在路径,$v$与$w$间存在路径,也即$u$与$v$间存在路径。\\
对图$G$中没有边相连的点$x$和$y$,在图$\overline{G}$中一定有边相连。
综上所述,对于非连通图$G$中任意两点,在图$\overline{G}$中必存在路径,所以非连通图$G$的补图$\overline{G}$是连通的。

\subsection*{Problem 1.63 离心率}



\end{document}