\documentclass[a4paper]{ctexart}
%\usepackage[]{ctex}
\usepackage[explicit]{titlesec}
\usepackage{xfrac}
\usepackage{marginnote}
\usepackage{fancybox}
\CTEXoptions[today=old]
%日期使用英文格式

%\renewcommand\thesection{1.\arabic{section}}
%暂时通过每次作业修改这个地方和标题来使得作业与题目Match。

% \titleformat{\section}{\normalfont\Large\bfseries}{#1\ \thesection}{1em}{}
% \titleformat{\subsection}{\normalfont\large\bfseries}{#1\ \thesubsection}{1em}{}
% \titleformat{\subsubsection}{\normalfont\normalsize\bfseries}{#1\ \thesubsubsection}{1em}{}
% \titleformat{\paragraph}{\normalfont\normalsize\bfseries}{#1\ \theparagraph}{1em}{}
% \titleformat{\subparagraph}{\normalfont\normalsize\bfseries}{#1\ \thesubparagraph}{1em}{}

% \titlespacing*{\chapter}{0pt}{50pt}{40pt}
% \titlespacing*{\section}{0pt}{3.5ex plus 1ex minus .2ex}{2.3ex plus .2ex}
% \titlespacing*{\subsection}{0pt}{3.25ex plus 1ex minus .2ex}{1.5ex plus .2ex}
% \titlespacing*{\subsubsection}{0pt}{3.25ex plus 1ex minus .2ex}{1.5ex plus .2ex}
% \titlespacing*{\paragraph}{0pt}{3.25ex plus 1ex minus .2ex}{1em}
% \titlespacing*{\subparagraph} {\parindent}{3.25ex plus 1ex minus .2ex}{1em}

\usepackage{fancyhdr}
\usepackage{extramarks}
\usepackage{amsmath}
%证明环境
\usepackage{amsthm}
%\usepackage{amsfonts}
%\usepackage{tikz}
%\usepackage[plain]{algorithm}
%\usepackage{algpseudocode}

%\usetikzlibrary{automata,positioning}

%
% Basic Document Settings
%

\topmargin=-0.5in
\evensidemargin=0in
\oddsidemargin=0in
\textwidth=6.5in
\textheight=10.0in
\headsep=0.25in
\marginparwidth = 48 pt
\reversemarginpar

\linespread{1.1}

\pagestyle{fancy}
\lhead{\hmwkAuthorName}
\chead{\hmwkClass\ (\hmwkClassInstructor): \hmwkTitle}
%\chead{\hmwkClass\ (\hmwkClassInstructor\ \hmwkClassTime): \hmwkTitle}
\rhead{\firstxmark}
\lfoot{\lastxmark}
\cfoot{\thepage}

\renewcommand\headrulewidth{0.4pt}
\renewcommand\footrulewidth{0.4pt}

\setlength\parindent{0pt}

\setcounter{secnumdepth}{3}
\newcounter{partCounter}
\newcounter{homeworkProblemCounter}
\setcounter{homeworkProblemCounter}{1}
%\nobreak\extramarks{Problem \arabic{homeworkProblemCounter}}{}\nobreak{}


%
% Homework Details
%   - Title
%   - Due date
%   - Class
%   - Section/Time
%   - Instructor
%   - Author
%

\newcommand{\hmwkTitle}{Homework\ 2}
\newcommand{\hmwkDueDate}{\today}
\newcommand{\hmwkClass}{计算机网络}
\newcommand{\hmwkClassTime}{}
\newcommand{\hmwkClassInstructor}{田臣}
\newcommand{\hmwkAuthorName}{\textbf{姓名}\:彭自远\quad\textbf{学号}\:171180631}
%\newcommand{\hmwkAuthorName}{\textbf{彭自远} \and \textbf{Davis Josh}}

%
% Title Page
%

\title{
    \vspace{2in}
    \textmd{\textbf{\hmwkClass:\ \hmwkTitle}}\\
    \normalsize\vspace{0.1in}\small{Printed\ on\ \hmwkDueDate}\\
    \vspace{0.1in}\large{\textit{\hmwkClassInstructor}}
    %\vspace{0.1in}\large{\textit{\hmwkClassInstructor\ \hmwkClassTime}}
    \vspace{3in}
}

\author{\hmwkAuthorName}
\date{}

\renewcommand{\part}[1]{\textbf{\large Part \Alph{partCounter}}\stepcounter{partCounter}\\}

\begin{document}

\maketitle

\pagebreak

\section*{Direct\ Link\ Networks:\ Textbook\ Charpter\ 5}

\subsection*{R4}
\paragraph*{Suppose two nodes start to transmit at the same time a packet of length L over a broadcast channel of rate R. Denote the propagation delay between the two nodes as $d_{prop}$ . Will there be a collision if $d_{prop}$ < $\frac{L}{R}$? Why or why not? \\
假设两个节点开始同时在速率R的广播信道上发送长度为L的分组。将两个节点之间的传播延迟表示为$d_{prop}$。 如果$d_{prop}$ <$\frac{L}{R}$会有冲突吗? 为什么或者为什么不?
}
\paragraph*{答:} 
%啥?

\subsection*{R5}
\paragraph*{In Section 5.3, we listed four desirable characteristics of a broadcast channel. Which of these characteristics does slotted ALOHA have? Which of these characteristics does token passing have?\\
在5.3节中,我们列出了广播频道的四个理想特征。 ALOHA有哪些特色? 令牌传递中有哪些特征?
}
\paragraph*{答:}  
ALOHA有这些特色:\\
\begin{itemize}
    \item 
\end{itemize}

令牌传递中有这些特征:\\
\begin{itemize}
    \item 
\end{itemize}
\subsection*{R6}
\paragraph*{In CSMA/CD, after the fifth collision, what is the probability that a node chooses K = 4? The result K = 4 corresponds to a delay of how many seconds?\\
在CSMA/CD中,在第五次碰撞之后,节点选择K = 4的概率是多少? 结果K=4对应于多少秒的延迟?}
\paragraph*{答:} 
\begin{itemize}
    \item 
\end{itemize}

\subsection*{R8}
\paragraph*{Why would the token-ring protocol be inefficient if a LAN had a very large perimeter?\\
如果局域网的周长非常大,为什么令牌环协议效率低下?}
\paragraph*{答:} 
\begin{itemize}
    \item 
\end{itemize}

\subsection*{P3}
\paragraph*{Suppose the information portion of a packet (D in Figure 5.3) contains 10 bytes consisting of the 8-bit unsigned binary ASCII representation of string “Networking.” Compute the Internet checksum for this data. \\
假设数据包的信息部分(图5.3中的D)包含10个字节,由字符串“Networking”的8位无符号二进制ASCII表示组成。计算此数据的Internet校验和。}
\paragraph*{答:} 
\begin{itemize}
    \item 
\end{itemize}

\subsection*{P5}
\paragraph*{Consider the 7-bit generator, G=10011, and suppose that D has the value 1010101010. What is the value of R?\\
考虑7位发生器,G = 10011,并假设D的值为1010101010.R的值是多少?}
\paragraph*{答:} 
\begin{itemize}
    \item 
\end{itemize}

\subsection*{P6}
\paragraph*{Consider the previous problem, but suppose that D has the value\\
考虑前一个问题,但假设D具有以下值\\
a. 1001010101.\\
b. 0101101010.\\
c. 1010100000.\\}
\paragraph*{答:} 
\begin{itemize}
    \item 
\end{itemize}

\subsection*{P8}
\paragraph*{In Section 5.3, we provided an outline of the derivation of the efficiency of slotted ALOHA. In this problem we’ll complete the derivation. \\
a. Recall that when there are N active nodes, the efficiency of slotted ALOHA is $Np(1 – p)^{N–1}$. Find the value of p that maximizes this expression.\\
b. Using the value of p found in (a), find the efficiency of slotted ALOHA by letting N approach infinity. Hint: $(1 -\frac{1}{N})^N$ approaches $\frac{1}{e}$ as N approaches infinity.\\
在5.3节中,我们提供了开槽ALOHA效率推导的概述。 在这个问题中,我们将完成推导。\\
a. 回想一下,当有N个活动节点时,时隙ALOHA的效率是$Np(1 – p)^{N–1}$。 找到最大化此表达式的p值。\\
b. 使用(a)中找到的p的值,通过让N接近无穷大来找到时隙ALOHA的效率。 提示:当1接近无穷大时,$(1 -\frac{1}{N})^N$接近$\frac{1}{e}$。
\\}
\paragraph*{答:} 
\begin{itemize}
    \item 
\end{itemize}

\subsection*{P10}
\paragraph*{Consider two nodes, A and B, that use the slotted ALOHA protocol to contend for a channel. Suppose node A has more data to transmit than node B, and node A’s retransmission probability $p_A$ is greater than node B’s retransmission probability, $p_B$.\\
a. Provide a formula for node A’s average throughput. What is the total efficiency of the protocol with these two nodes?\\
b. If $p_A$ = $2p_B$ , is node A’s average throughput twice as large as that of node B? Why or why not? If not, how can you choose p A and p B to make that happen?\\
c. In general, suppose there are N nodes, among which node A has retransmission probability 2p and all other nodes have retransmission probability p. Provide expressions to compute the average throughputs of node A and of any other node.\\
考虑两个节点A和B,它们使用时隙ALOHA协议来竞争信道。 假设节点A比节点B有更多数据要传输,节点A的重传概率$ p_A $大于节点B的重传概率$ p_B $。\\
a. 为节点A的平均吞吐量提供公式。 这两个节点的协议总效率是多少?\\
b. 如果$ p_A $ = $ 2p_B $,节点A的平均吞吐量是节点B的两倍吗? 为什么或者为什么不? 如果没有,你如何选择p A和p B来实现这一目标?
c. 一般来说,假设有N个节点,其中节点A具有重传概率2p,所有其他节点具有重传概率p. 提供表达式来计算节点A和任何其他节点的平均吞吐量。}
\paragraph*{答:} 
\begin{itemize}
    \item 
\end{itemize}

\subsection*{P18}
\paragraph*{Suppose nodes A and B are on the same 10 Mbps broadcast channel, and the propagation delay between the two nodes is 325 bit times. Suppose CSMA/CD and Ethernet packets are used for this broadcast channel. Suppose node A begins transmitting a frame and, before it finishes, node B begins transmitting a frame. Can A finish transmitting before it detects that B has transmitted? Why or why not? If the answer is yes, then A incorrectly believes that its frame was successfully transmitted without a collision. Hint: Suppose at time t = 0 bits, A begins transmitting a frame. In the worst case, A transmits a minimum-sized frame of 512 + 64 bit times. So A would finish transmitting the frame at t = 512 + 64 bit times. Thus, the answer is no, if B’s signal reaches A before bit time t = 512 + 64 bits. In the worst case, when does B’s signal reach A?\\}
\paragraph*{答:} 
\begin{itemize}
    \item 
\end{itemize}

\subsection*{P19}
\paragraph*{Suppose nodes A and B are on the same 10 Mbps broadcast channel, and the propagation delay between the two nodes is 245 bit times. Suppose A and B send Ethernet frames at the same time, the frames collide, and then A and B choose different values of K in the CSMA/CD algorithm. Assuming no other nodes are active, can the retransmissions from A and B collide? For our purposes, it suffices to work out the following example. Suppose A and B begin transmission at t = 0 bit times. They both detect collisions at t = 245 bit times. Suppose K A = 0 and K B = 1. At what time does B schedule its retransmission? At what time does A begin transmission? (Note: The nodes must wait for an idle channel after returning to Step 2—see protocol.) At what time does A’s signal reach B? Does B refrain from transmitting at its scheduled time?\\}
\paragraph*{答:} 
\begin{itemize}
    \item 
\end{itemize}

\subsection*{P23}
\paragraph*{Consider Figure 5.15. Suppose that all links are 100 Mbps. What is the maximum total aggregate throughput that can be achieved among the 9 hosts and 2 servers in this network? You can assume that any host or serrver can send to any other host or server. Why?\\}
\paragraph*{答:} 
\begin{itemize}
    \item 
\end{itemize}

\subsection*{P24}
\paragraph*{Suppose the three departmental switches in Figure 5.15 are replaced by hubs. All links are 100 Mbps. Now answer the questions posed in problem P23.\\}
\paragraph*{答:} 
\begin{itemize}
    \item 
\end{itemize}

\subsection*{P25}
\paragraph*{Why would the token-ring protocol be inefficient if a LAN had a very large\\}
\paragraph*{答:} 
\begin{itemize}
    \item 
\end{itemize}

\subsection*{P26}
\paragraph*{Why would the token-ring protocol be inefficient if a LAN had a very large\\}
\paragraph*{答:} 
\begin{itemize}
    \item 
\end{itemize}

\section*{Direct\ Link\ Networks:\ Textbook\ Charpter\ 6}
\subsection*{R7}
\paragraph*{Why would the token-ring protocol be inefficient if a LAN had a very large\\}
\paragraph*{答:} 
\begin{itemize}
    \item 
\end{itemize}

\subsection*{P5}
\paragraph*{Why would the token-ring protocol be inefficient if a LAN had a very large\\}
\paragraph*{答:} 
\begin{itemize}
    \item 
\end{itemize}

\subsection*{P6}
\paragraph*{Why would the token-ring protocol be inefficient if a LAN had a very large\\}
\paragraph*{答:} 
\begin{itemize}
    \item 
\end{itemize}

\subsection*{P8}
\paragraph*{Why would the token-ring protocol be inefficient if a LAN had a very large\\}
\paragraph*{答:} 
\begin{itemize}
    \item 
\end{itemize}


\end{document}