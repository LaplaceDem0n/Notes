\documentclass[a4paper]{ctexart}
%\usepackage[]{ctex}
\usepackage[explicit]{titlesec}
\usepackage{xfrac}
\usepackage{marginnote}
\usepackage{fancybox}
\CTEXoptions[today=old]
%日期使用英文格式

%\renewcommand\thesection{1.\arabic{section}}
%暂时通过每次作业修改这个地方和标题来使得作业与题目Match。

% \titleformat{\section}{\normalfont\Large\bfseries}{#1\ \thesection}{1em}{}
% \titleformat{\subsection}{\normalfont\large\bfseries}{#1\ \thesubsection}{1em}{}
% \titleformat{\subsubsection}{\normalfont\normalsize\bfseries}{#1\ \thesubsubsection}{1em}{}
% \titleformat{\paragraph}{\normalfont\normalsize\bfseries}{#1\ \theparagraph}{1em}{}
% \titleformat{\subparagraph}{\normalfont\normalsize\bfseries}{#1\ \thesubparagraph}{1em}{}

% \titlespacing*{\chapter}{0pt}{50pt}{40pt}
% \titlespacing*{\section}{0pt}{3.5ex plus 1ex minus .2ex}{2.3ex plus .2ex}
% \titlespacing*{\subsection}{0pt}{3.25ex plus 1ex minus .2ex}{1.5ex plus .2ex}
% \titlespacing*{\subsubsection}{0pt}{3.25ex plus 1ex minus .2ex}{1.5ex plus .2ex}
% \titlespacing*{\paragraph}{0pt}{3.25ex plus 1ex minus .2ex}{1em}
% \titlespacing*{\subparagraph} {\parindent}{3.25ex plus 1ex minus .2ex}{1em}

\usepackage{fancyhdr}
\usepackage{extramarks}
\usepackage{amsmath}
%证明环境
\usepackage{amsthm}
%\usepackage{amsfonts}
%\usepackage{tikz}
%\usepackage[plain]{algorithm}
%\usepackage{algpseudocode}

%\usetikzlibrary{automata,positioning}

%
% Basic Document Settings
%

\topmargin=-0.5in
\evensidemargin=0in
\oddsidemargin=0in
\textwidth=6.5in
\textheight=10.0in
\headsep=0.25in
\marginparwidth = 48 pt
\reversemarginpar

\linespread{1.1}

\pagestyle{fancy}
\lhead{\hmwkAuthorName}
\chead{\hmwkClass\ (\hmwkClassInstructor): \hmwkTitle}
%\chead{\hmwkClass\ (\hmwkClassInstructor\ \hmwkClassTime): \hmwkTitle}
\rhead{\firstxmark}
\lfoot{\lastxmark}
\cfoot{\thepage}

\renewcommand\headrulewidth{0.4pt}
\renewcommand\footrulewidth{0.4pt}

\setlength\parindent{0pt}

\setcounter{secnumdepth}{3}
\newcounter{partCounter}
\newcounter{homeworkProblemCounter}
\setcounter{homeworkProblemCounter}{1}
%\nobreak\extramarks{Problem \arabic{homeworkProblemCounter}}{}\nobreak{}


%
% Homework Details
%   - Title
%   - Due date
%   - Class
%   - Section/Time
%   - Instructor
%   - Author
%

\newcommand{\hmwkTitle}{Homework\ 2}
\newcommand{\hmwkDueDate}{\today}
\newcommand{\hmwkClass}{计算机网络}
\newcommand{\hmwkClassTime}{}
\newcommand{\hmwkClassInstructor}{田臣}
\newcommand{\hmwkAuthorName}{\textbf{姓名}\:彭自远\quad\textbf{学号}\:171180631}
%\newcommand{\hmwkAuthorName}{\textbf{彭自远} \and \textbf{Davis Josh}}

%
% Title Page
%

\title{
    \vspace{2in}
    \textmd{\textbf{\hmwkClass:\ \hmwkTitle}}\\
    \normalsize\vspace{0.1in}\small{Printed\ on\ \hmwkDueDate}\\
    \vspace{0.1in}\large{\textit{\hmwkClassInstructor}}
    %\vspace{0.1in}\large{\textit{\hmwkClassInstructor\ \hmwkClassTime}}
    \vspace{3in}
}

\author{\hmwkAuthorName}
\date{}

\renewcommand{\part}[1]{\textbf{\large Part \Alph{partCounter}}\stepcounter{partCounter}\\}

\begin{document}

\maketitle

\pagebreak

\section*{Direct\ Link\ Networks:\ Textbook\ Charpter\ 5}

\subsection*{R4}
\paragraph*{Suppose two nodes start to transmit at the same time a packet of length L over a broadcast channel of rate R. Denote the propagation delay between the two nodes as $d_{prop}$ . Will there be a collision if $d_{prop}$ < $\frac{L}{R}$? Why or why not? \\
假设两个节点开始同时在速率R的广播信道上发送长度为L的分组。将两个节点之间的传播延迟表示为$d_{prop}$。 如果$d_{prop}$ <$\frac{L}{R}$会有冲突吗? 为什么或者为什么不?
}
\paragraph*{答:} 
%啥?

\subsection*{R5}
\paragraph*{In Section 5.3, we listed four desirable characteristics of a broadcast channel. Which of these characteristics does slotted ALOHA have? Which of these characteristics does token passing have?\\
在5.3节中,我们列出了广播频道的四个理想特征。 ALOHA有哪些特色? 令牌传递中有哪些特征?
}
\paragraph*{答:}  
ALOHA有这些特色:\\
\begin{itemize}
    \item 
\end{itemize}

令牌传递中有这些特征:\\
\begin{itemize}
    \item 
\end{itemize}
\subsection*{R6}
\paragraph*{In CSMA/CD, after the fifth collision, what is the probability that a node chooses K = 4? The result K = 4 corresponds to a delay of how many seconds?\\
在CSMA/CD中,在第五次碰撞之后,节点选择K = 4的概率是多少? 结果K=4对应于多少秒的延迟?}
\paragraph*{答:} 
\begin{itemize}
    \item 
\end{itemize}

\subsection*{R8}
\paragraph*{Why would the token-ring protocol be inefficient if a LAN had a very large perimeter?\\
如果局域网的周长非常大,为什么令牌环协议效率低下?}
\paragraph*{答:} 
\begin{itemize}
    \item 
\end{itemize}

\subsection*{P3}
\paragraph*{Suppose the information portion of a packet (D in Figure 5.3) contains 10 bytes consisting of the 8-bit unsigned binary ASCII representation of string “Networking.” Compute the Internet checksum for this data. \\
假设数据包的信息部分(图5.3中的D)包含10个字节,由字符串“Networking”的8位无符号二进制ASCII表示组成。计算此数据的Internet校验和。}
\paragraph*{答:} 
\begin{itemize}
    \item 
\end{itemize}

\subsection*{P5}
\paragraph*{Consider the 7-bit generator, G=10011, and suppose that D has the value 1010101010. What is the value of R?\\
考虑7位发生器,G = 10011,并假设D的值为1010101010.R的值是多少?}
\paragraph*{答:} 
\begin{itemize}
    \item 
\end{itemize}

\subsection*{P6}
\paragraph*{Consider the previous problem, but suppose that D has the value\\
考虑前一个问题,但假设D具有以下值\\
a. 1001010101.\\
b. 0101101010.\\
c. 1010100000.\\}
\paragraph*{答:} 
\begin{itemize}
    \item 
\end{itemize}

\subsection*{P8}
\paragraph*{In Section 5.3, we provided an outline of the derivation of the efficiency of slotted ALOHA. In this problem we’ll complete the derivation. \\
a. Recall that when there are N active nodes, the efficiency of slotted ALOHA is $Np(1 – p)^{N–1}$. Find the value of p that maximizes this expression.\\
b. Using the value of p found in (a), find the efficiency of slotted ALOHA by letting N approach infinity. Hint: $(1 -\frac{1}{N})^N$ approaches $\frac{1}{e}$ as N approaches infinity.\\
在5.3节中,我们提供了开槽ALOHA效率推导的概述。 在这个问题中,我们将完成推导。\\
a. 回想一下,当有N个活动节点时,时隙ALOHA的效率是$Np(1 – p)^{N–1}$。 找到最大化此表达式的p值。\\
b. 使用(a)中找到的p的值,通过让N接近无穷大来找到时隙ALOHA的效率。 提示:当1接近无穷大时,$(1 -\frac{1}{N})^N$接近$\frac{1}{e}$。
\\}
\paragraph*{答:} 
\begin{itemize}
    \item 
\end{itemize}

\subsection*{P10}
\paragraph*{Consider two nodes, A and B, that use the slotted ALOHA protocol to contend for a channel. Suppose node A has more data to transmit than node B, and node A’s retransmission probability $p_A$ is greater than node B’s retransmission probability, $p_B$.\\
a. Provide a formula for node A’s average throughput. What is the total efficiency of the protocol with these two nodes?\\
b. If $p_A$ = $2p_B$ , is node A’s average throughput twice as large as that of node B? Why or why not? If not, how can you choose p A and p B to make that happen?\\
c. In general, suppose there are N nodes, among which node A has retransmission probability 2p and all other nodes have retransmission probability p. Provide expressions to compute the average throughputs of node A and of any other node.\\
考虑两个节点A和B,它们使用时隙ALOHA协议来竞争信道。 假设节点A比节点B有更多数据要传输,节点A的重传概率$ p_A $大于节点B的重传概率$ p_B $。\\
a. 为节点A的平均吞吐量提供公式。 这两个节点的协议总效率是多少?\\
b. 如果$ p_A $ = $ 2p_B $,节点A的平均吞吐量是节点B的两倍吗? 为什么或者为什么不? 如果没有,你如何选择p A和p B来实现这一目标?
c. 一般来说,假设有N个节点,其中节点A具有重传概率2p,所有其他节点具有重传概率p. 提供表达式来计算节点A和任何其他节点的平均吞吐量。}
\paragraph*{答:} 
\begin{itemize}
    \item 
\end{itemize}

\marginnote{写到这里发现其实第七版中文教材与第六版英文教材的习题题号一致只有章节不同以下略去中文翻译}

\subsection*{P18}
\paragraph*{Suppose nodes A and B are on the same 10 Mbps broadcast channel, and the propagation delay between the two nodes is 325 bit times. Suppose CSMA/CD and Ethernet packets are used for this broadcast channel. Suppose node A begins transmitting a frame and, before it finishes, node B begins transmitting a frame. Can A finish transmitting before it detects that B has transmitted? Why or why not? If the answer is yes, then A incorrectly believes that its frame was successfully transmitted without a collision. Hint: Suppose at time t = 0 bits, A begins transmitting a frame. In the worst case, A transmits a minimum-sized frame of 512 + 64 bit times. So A would finish transmitting the frame at t = 512 + 64 bit times. Thus, the answer is no, if B’s signal reaches A before bit time t = 512 + 64 bits. In the worst case, when does B’s signal reach A?\\}
\paragraph*{答:} 
\begin{itemize}
    \item 
\end{itemize}

\subsection*{P19}
\paragraph*{Suppose nodes A and B are on the same 10 Mbps broadcast channel, and the propagation delay between the two nodes is 245 bit times. Suppose A and B send Ethernet frames at the same time, the frames collide, and then A and B choose different values of K in the CSMA/CD algorithm. Assuming no other nodes are active, can the retransmissions from A and B collide? For our purposes, it suffices to work out the following example. Suppose A and B begin transmission at t = 0 bit times. They both detect collisions at t = 245 bit times. Suppose K A = 0 and K B = 1. At what time does B schedule its retransmission? At what time does A begin transmission? (Note: The nodes must wait for an idle channel after returning to Step 2—see protocol.) At what time does A’s signal reach B? Does B refrain from transmitting at its scheduled time?\\}
\paragraph*{答:} 
\begin{itemize}
    \item 
\end{itemize}

\subsection*{P23}
\paragraph*{Consider Figure 5.15. Suppose that all links are 100 Mbps. What is the maximum total aggregate throughput that can be achieved among the 9 hosts and 2 servers in this network? You can assume that any host or serrver can send to any other host or server. Why?\\}
\paragraph*{答:} 
\begin{itemize}
    \item 
\end{itemize}

\subsection*{P24}
\paragraph*{Suppose the three departmental switches in Figure 5.15 are replaced by hubs. All links are 100 Mbps. Now answer the questions posed in problem P23.\\}
\paragraph*{答:} 
\begin{itemize}
    \item 
\end{itemize}

\subsection*{P25}
\paragraph*{Suppose that all the switches in Figure 5.15 are replaced by hubs. All links are 100 Mbps. Now answer the questions posed in problem P23. \\}
\paragraph*{答:} 
\begin{itemize}
    \item 
\end{itemize}

\subsection*{P26}
\paragraph*{Let’s consider the operation of a learning switch in the context of a network in which 6 nodes labeled A through F are star connected into an Ethernet switch. Suppose that\\
(i) B sends a frame to E,\\
(ii) E replies with a frame to B,\\
(iii) A sends a frame to B, (iv) B replies with a frame to A. The switch table is initially empty. Show the state of the switch table before and after each of these events. For each of these events, identify the link(s) on which the transmitted frame will be forwarded, and briefly justify your answers.\\}
\paragraph*{答:} 
\begin{itemize}
    \item 
\end{itemize}

\section*{Direct\ Link\ Networks:\ Textbook\ Charpter\ 6}
\subsection*{R7}
\paragraph*{R7. Why are acknowledgments used in 802.11 but not in wired Ethernet?\\}
\paragraph*{答:} 
\begin{itemize}
    \item 
\end{itemize}

\subsection*{P5}
\paragraph*{Suppose there are two ISPs providing WiFi access in a particular café, with each ISP operating its own AP and having its own IP address block.\\
a. Further suppose that by accident, each ISP has configured its AP to operate over channel 11. Will the 802.11 protocol completely break down in this situation? Discuss what happens when two stations, each associated with a different ISP, attempt to transmit at the same time.\\
b. Now suppose that one AP operates over channel 1 and the other over channel 11. How do your answers change? \\}
\paragraph*{答:} 
\begin{itemize}
    \item 
\end{itemize}

\subsection*{P6}
\paragraph*{In step 4 of the CSMA/CA protocol, a station that successfully transmits a frame begins the CSMA/CA protocol for a second frame at step 2, rather than at step 1. What rationale might the designers of CSMA/CA have had in mind by having such a station not transmit the second frame immediately (if the
channel is sensed idle)?\\}
\paragraph*{答:} 
\begin{itemize}
    \item 
\end{itemize}

\subsection*{P8}
\paragraph*{Consider the scenario shown in Figure 6.33, in which there are four wireless nodes, A, B, C, and D. The radio coverage of the four nodes is shown via the shaded ovals; all nodes share the same frequency. When A transmits, it can only be heard/received by B; when B transmits, both A and C can hear/receive from B; when C transmits, both B and D can hear/receive from C; when D transmits, only C can hear/receive from D. Suppose now that each node has an infinite supply of messages that it wantsto send to each of the other nodes. If a message’s destination is not an immediate neighbor, then the message must be relayed. For example, if A wants to send to D, a message from A must first be sent to B, which then sends the message to C, which then sends the message to D. Time is slotted, with a message transmission time taking exactly one time slot, e.g., as in slotted Aloha. During a slot, a node can do one of the following: \\
(i) send a message;\\
(ii) receive a message (if exactly one message is being sent to it),\\
(iii) remain silent. As always, if a node hears two or more simultaneous transmissions, a collision occurs and none of the transmitted messages are received successfully. You can assume here that there are no bit-level errors, and thus if exactly one message is sent, it will be received correctly by those within the\\}
\subparagraph*{a. Suppose now that an omniscient controller (i.e., a controller that knows the state of every node in the network) can command each node to do whatever it (the omniscient controller) wishes, i.e., to send a message, to receive a message, or to remain silent. Given this omniscient controller, what is the maximum rate at which a data message can be transferred from C to A, given that there are no other messages between any other source/destination pairs?}
\paragraph*{答:} 
\begin{itemize}
    \item 
\end{itemize}
\subparagraph*{b. Suppose now that A sends messages to B, and D sends messages to C. What is the combined maximum rate at which data messages can flow from A to B and from D to C?}
\paragraph*{答:} 
\begin{itemize}
    \item 
\end{itemize}
\subparagraph*{c. Suppose now that A sends messages to B, and C sends messages to D.Suppose now that A sends messages to B, and C sends messages to D.
What is the combined maximum rate at which data messages can flow from A to B and from C to D?}
\paragraph*{答:} 
\begin{itemize}
    \item 
\end{itemize}
\subparagraph*{d. Suppose now that the wireless links are replaced by wired links. Repeat questions (a) through (c) again in this wired scenario.}
\paragraph*{答:} 
\begin{itemize}
    \item 
\end{itemize}
\subparagraph*{e. Now suppose we are again in the wireless scenario, and that for every data message sent from source to destination, the destination will send an ACK message back to the source (e.g., as in TCP). Also suppose that each ACK message takes up one slot. Repeat questions (a) – (c) above for this scenario.} 
\paragraph*{答:} 
\begin{itemize}
    \item 
\end{itemize}


\end{document}